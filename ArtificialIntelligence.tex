\documentclass{article}
\usepackage{graphicx}
\usepackage[top=1in, bottom=1in, left=1.25in, right=1.25in]{geometry}
\usepackage{multicol}
\usepackage{cite}
\usepackage{ctex}
\usepackage[justification=centering]{caption}
\usepackage[colorlinks,
            linkcolor=red,
            anchorcolor=blue,
            citecolor=green,
            backref=page
            ]{hyperref}
\bibliographystyle{plain}
\title{Artificial Intelligence}
\author{Fangfang Li}
\date{May 27, 2018}
\begin{document}
\twocolumn
\maketitle
\section{Introduction}

 AI refers to artificial intelligence which is a little bit difficult for unprofessional persons to understand. With the development of the modern science, AI is becoming its foundation which is related to nearly all industrial circles. It includes the imitation on human way of thinking and the style of living, being applied in industry, economy, medicine, education, science research and so on. Commonly, it is believed that robots are the presentation of the AI technology which are used everywhere. But actually, we can find the AI technology in our cars, mobile phones, washing-machine, computers and even our cookers. Perhaps, it is difficult for us to understand its scientific term, but it is convenient for us to use it. Surely enough, AI is a kind of high-tech, we should say that we cannot live any better withou AI technology. Hopefully, our life will become more enjoyable, easy and comfortable with the help of AI.

\section{Artificial Intelligence}
Artificial Intelligence is usually defined as the science of making computers do things that require intelligence when done by humans, as shown in Figure~\ref{1}. AI has had some success in limited, or simplified, domains. However, the five decades since the inception of AI have brought only very slow progress, and early optimism concerning the attainment of human-level intelligence has given way to an appreciation of the profound difficulty of the problem.
The area of computer science focuses on creating machines that can engage on behaviors that humans consider intelligent. The ability to create intelligent machines has intrigued humans since ancient times, and today with the advent of the computer and 50 years of research into AI programming techniques, the dream of smart machines is becoming a reality. Researchers are creating systems which can mimic human thought, understand speech, beat the best human chessplayer, and countless other feats never before possible. Find out how the military is applying AI logic to its hi-tech systems, and how in the near future Artificial Intelligence may impact our lives~\cite{Lawrence1997Face}.

\begin{figure}[!htb]
\centering
\includegraphics[width=0.4\textwidth]{A.jpg}
\caption{Artificial Intelligence. }
\label{1}
\end{figure}


\bibliography{Book1}
\end{document}