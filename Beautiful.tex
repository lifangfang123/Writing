\documentclass{article}
\usepackage{graphicx}
\usepackage{subfigure}
\usepackage{cite}
\bibliographystyle{plain}

\newcommand{\upcite}[1]{\textsuperscript{\textsuperscript{\cite{#1}}}}
\title{ Genuine compliment.}
\author{Fangfang Li}
\date{\today}
\begin{document}
\maketitle
As research shows, interpersonal warmth explains the self-fulfilling prophecy of anticipated acceptance; study participants who expected to be accepted were perceived as more likable. As the figure \ref{1} shows.  All of which sounds great, but the trick, when you're shy or insecure, is actually believing that other people will like you. When you're in an unfamiliar setting or an uncomfortable position, it's a lot easier to assume people won't like you. So how can you convince yourself that people will like you? Positive self-talk won't cut it. Instead, close your eyes, take a deep breath, and commit to taking a few steps that ensure almost anyone will like you.

\begin{figure}[htbp]
\centering
\includegraphics[width=0.5\textwidth]{L.jpg}
\caption{Beautiful}
\label{1}
\end{figure}

\par Everyone loves to be praised, especially since no one gets enough praise. Show interest by asking questions.\cite{higham1994bibtex} But go past, "What do you do?" Ask what it's like to do what the person does. Ask what's hard about it. Ask what the person loves about it. You'll soon find things to compliment.
\renewcommand\refname{Reference}

\bibliography{bibfile}
\end{document}
