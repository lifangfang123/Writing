\documentclass[a4paper]{article}
\usepackage{graphicx,float}
\usepackage{subfigure}
\usepackage[justification=centering]{caption}
\usepackage{cite}
\usepackage{color}
\setlength{\parindent}{1em}
\bibliographystyle{plain}
\newcommand{\upcite}[1]{\textsuperscript{\textsuperscript{\cite{#1}}}}
 \begin{document}
 \title{Automatic Analysis of Facial Affect}
\author{Fangfang Li}
\date{\today}
\maketitle
 \par In this paper, they break down facial affect recognition systems into their fundamental components : facial registration, representation, dimensionality reduction and recognition. They discuss the role of each component in dealing with the challenges in affect recognition. They ana-lyse facial representations in detail by discussing their advantages and limitations, the type of information they encode, their ability to recognise subtle expressions, their dimensionality and computational complexity. They further discuss new classifiers and statistical models that exploit affect-specific dynamics by modelling the temporal varia-tion of emotions or expressions, the statistical dependencies among different facial actions and the influence of person-specific cues in facial appearance.\cite{higham1994bibtex} They review evaluation procedures and metrics, and analyse the outcome of recently organised automatic affect recognition competi-tions. Finally,They discuss open issues and list potential future directions. The figure \ref{1} {\color{red} {deep learning}}.
\begin{figure}[!htp]
\centering
\includegraphics[width=0.5\textwidth]{6.jpg}
\caption{{\color{red} {deep learning}}}
\label{1}
\end{figure}
\renewcommand\refname{Reference}
\bibliography{bibfile}
\end{document}
