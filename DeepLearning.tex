\documentclass{article}
\usepackage{graphicx}
\usepackage[top=1in, bottom=1in, left=1.25in, right=1.25in]{geometry}
\usepackage{multicol}
\usepackage{cite}
\usepackage{ctex}
\usepackage[justification=centering]{caption}
\usepackage[colorlinks,citecolor=green]{hyperref}
\bibliographystyle{plain}
\title{Deep learning}
\author{Fangfang Li}
\date{May 23,2018}
\begin{document}
\twocolumn
\maketitle

\begin{abstract}

 \emph{Deep learning is a multilayer neural network architecture, data from a large study in the real world, all kinds of things that can be directly used for representation of computer calculations, it is considered to be intelligent machines possible ��brain structure"}
\end{abstract}
\section{Introduction}
\par In recent years, there��s been a resurgence in the field of Artificial Intelligence~\cite{Goossens2000}. It��s spread beyond the academic world with major players like Google, Microsoft, and?Facebook creating their own research teams and making some impressive acquisition. Some this can be attributed to the abundance of raw data generated by social network users, much of which needs to be analyzed, as well as to the cheap computational power available via GPGUs. But beyond these phenomena, this resurgence has been powered in no small part by a new trend in AI, specifically in machine learning, known as ��Deep Learning��~\cite{Yong2017}. One of the earliest supervised training algorithms is that of the perceptron, a basic neural network building block.
Say we have points in the plane, as shown in Figure~\ref{1}. We��re given a new point and we want to guess its label (this is akin to the ��Dog�� and ��Not dog�� scenario above). How do we do it? One approach might be to look at the closest neighbor and return that point��s label. But a slightly more intelligent way of going about it would be to pick a line that best separates the labeled data and use that as your classifier.The process of neural networks can be expressed as:
\begin{equation}a_{j}^{l}=\sigma\left(\sum_{k} w_{jk}^{l}a_{k}^{l-1}+b_{j}^{l}\right)\end{equation}
\begin{figure}[!htb]
\centering
\includegraphics[width=0.4\textwidth]{D.png}
\caption{ Deep learning}
\label{1}
\end{figure}

\bibliography{bibfile}
\end{document}