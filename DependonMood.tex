\documentclass{article}
\thispagestyle{plain}
\linespread{1.5}
\begin{document}
\begin{center}
{\bfseries \LARGE Is Mona Lisa smiling? }
\end{center}
\begin{center}
Fangfang Li
\end{center}
\begin{center}
\today
\end{center}
\par Scientists have discovered why the Mona Lisa's expression looks so different to different people and at different times. For centuries, art lovers and critics have been perplexed by and debated the Leonardo Da Vinci paintings gaze and slight smile - or is it a grimace? But new research from the University of California, San Francisco has shed new light on the luminous and seemingly changing face of the Mona Lisa. Through experiments on visual perception and neurology, they discovered that our emotions really do alter how we see a neutral face. So, she and her team predicted that how we perceive a new face - as happy, sad, friendly, neutral - actually has a lot more to do with the feelings we are carrying around when we greet it than the expression on that face. We all have one dominant eye and one more passive non-dominant one. If each eye is receiving different information, we only consciously perceive what dominant one sees. But non-dominant sights can still seep into our subconcscious. This relies on the modern theory of 'the brain as a predictive organ, instead of a reactive one,' says Dr Siegel. In other words, we have a lifetime of experience and we use those experiences to predict what we are going to experience next. 
\footnote{from"China Daily"}
\begin{thebibliography}{99}
\bibitem{pa}Jane Raymond.~Interactions of attention,~emotion and motivation~[J]~.~ Progress in Brain Research.~2009~(3):~293~-~308.
\end{thebibliography}
\end{document}