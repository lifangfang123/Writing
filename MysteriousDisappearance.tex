\documentclass{article}
\thispagestyle{plain}
\linespread{1.5}
\begin{document}
\begin{center}
{\bfseries \LARGE Mysterious disappearance}
\end{center}
\begin{center}
Fangfang Li
\end{center}
\begin{center}
\today
\end{center}
\par We start with the granddaddy of all mysterious locations, the Bermuda Triangle. This stretch of ocean, spanning from South Florida to Puerto Rico to Bermuda, has been the site of multiple disappearances over the years, including the vanishing of the USS Cyclops with 309 crewmen aboard in 1918. Both planes and ships have gone missing, never to be recovered. Not as famous as its southern cousin, the Michigan Triangle extends between Michigan and Wisconsin, and has been the location of a number of prominent disappearances. Yet another triangle, though this one is landlocked. The Bennington Triangle, located in southwestern Vermont, marks the last known location of five people who went missing between 1945 and 1950. Our final triangular entry on the list, the Nevada Triangle marks a 25,000 square mile stretch of desert and mountains in the state of Nevada. This desolate stretch is in the vicinity of the notorious Area 51 military zone, and has been the site of a number of disappearances. Stretching back in time to the beginning of European colonization of the New World, Roanoke is one of the earliest recorded mysterious disappearances.
\footnote{from"China Daily"}
\begin{thebibliography}{99}
\bibitem{pa} Jane Raymond.~Emotion in Spreaking and Singing~[J].~SJGT Thieme.~2004~(03):~146~-~149.
\end{thebibliography}
\end{document}