\documentclass[twocolumn]{article}
\usepackage{cite}
\usepackage{graphicx}
\bibliographystyle{plain}
\newcommand{\upcite}[1]{\textsuperscript{\textsuperscript{\cite{#1}}}}
\begin{document}
\title{Talking}
\author{Fangfang Li\\[4pt]
Department of Electronic Engineering,Ocean University of China}
\date{May,5,2018}
\maketitle
\begin{abstract}
Are you talking to someone and it's hard to see what the other person's eyes say? People often think that avoiding eye contact when talking is a sign of fear or insecurity. However, scientists from Kyoto University in JaAre you talking to someone and it's hard to see what the other person's eyes say? People often think that avoiding eye contact when talking is a sign of fear or insecurity. However, scientists from Kyoto University in Japan gave another scientific explanation.pan gave another scientific explanation.
\end{abstract}
\section{Intoduction}
You can either travel or read, but either your body or soul must be on the way. The popular saying has inspired many people to read or go sightseeing. Are you planning a trip to relax after working so hard for such a long time? Traveling, just like reading, is a refreshing journey, a temporary retreat from the bustling world. Here are 3 books we recommend that you take on your trip.
\begin{figure}[htbp]
%\small
\centering
\includegraphics[width=0.5\textwidth]{eye.jpg}
\caption{Eye}
\label{1}
\end{figure}
\par It turns out we're not just awkward, our brains actually can't handle the tasks of thinking of the right words and focusing on a face at the same time.\upcite{higham1994bibtex} The effect becomes more noticeable when someone is trying to come up with less familiar words, which is thought to use the same mental resources as sustaining eye contact. Scientists from Kyoto University in Japan put this to the test in 2016 by having 26 volunteers play word association games while staring at computer-generated faces. When making eye contact, the participants found it harder to come up with links between words. "Although eye contact and verbal processing appear independent, people frequently avert their eyes from interlocutors during conversation," wrote
\renewcommand\refname{Reference}
%\bibliographystyle{plain}
\bibliography{bibfile}
\footnote{from"China Daily"}
\end{document}
